\documentclass[10pt]{beamer}
\usetheme{metropolis}

\usepackage{amsmath, booktabs, fontawesome5, natbib, subfigure, xcolor}

\usepackage[font=small,skip=0pt]{caption}

\usepackage{pgfplots}
\usepgfplotslibrary{dateplot}

\usepackage{xspace}
\newcommand{\themename}{\textbf{\textsc{metropolis}}\xspace}

\usepackage{graphicx}
\graphicspath{{../imgs/}}

\usepackage[T1]{fontenc}
\usepackage[french]{babel}

\usepackage{appendixnumberbeamer}

\usepackage{tikz}
\usetikzlibrary{shapes.geometric, arrows}
\tikzstyle{rect} = [rectangle, rounded corners, minimum width=2cm, minimum height=1.5cm, text centered, draw=black, fill=black!30]
\tikzstyle{squa} = [square, rounded corners, minimum width=1.25cm, minimum height=1.25cm, text centered, draw=black, fill=black!30]
\tikzstyle{elli} = [ellipse, minimum width=1.25cm, minimum height=1cm, text centered, draw=black, fill=black!30]
\tikzstyle{circ} = [circle, minimum width=1cm, minimum height=1cm, text centered, draw=black, fill=black!30]
\tikzstyle{arrow} = [thick,->,>=stealth]
\tikzstyle{drrow} = [thick,<->,>=stealth]
\tikzstyle{dline} = [dashed, ->, >=stealth]
\tikzstyle{dotted} = [densely dotted, ->, >=stealth]

\def\firstcircle{(90:1.75cm) circle (2.5cm)}
\def\secondcircle{(210:1.75cm) circle (2.5cm)}
\def\thirdcircle{(330:1.75cm) circle (2.5cm)}
\tikzset{fontscale/.style = {font=\relsize{#1}}}

\titlegraphic{\hfill\includegraphics[height=1.25cm]{~/Documents/scpobx/logo.pdf}}
\title{Méthodes des sciences sociales}
\subtitle{Séance 2: L'objet de recherche}
\author{Mickael Temporão}
% \institute{Sciences Po Bordeaux}
\date{}

\begin{document}

\maketitle

\begin{frame}{Méthodes des sciences sociales}
    \onslide<1->{
        \begin{block}{Ordre du jour}
            \begin{itemize}
                \item<2-> Présentation et discussion
                \item<3-> La methode scientifique et ses alternatives
                \item<5-> Le projet de recherche
            \end{itemize}
        \end{block}
    }
    \vspace{24pt}
    \onslide<6->{
        \begin{block}{\faTrophy~BONUS}
        \begin{itemize}
            \item[] \faGithubSquare~Github
            \item[] \faMarkdown~R
            \item[] \faPollH~Rétrospective
        \end{itemize}
        \end{block}
    }
\end{frame}

\section{Présentation et discussion}
\section{La méthode scientifique}


\begin{frame}{La méthode scientifique}
    \onslide<1-> {
        \begin{block}{Qu'est-ce que la science?}
            \begin{itemize}
                \item <2-> Un pratique utilisant une méthode systématique d'observation de phénomènes dans le but d'accroitre notre connaissance sur un sujet.
                \item <3-> La méthode systématique = la méthode scientifique.
            \end{itemize}
        \end{block}
    }
    \onslide<4-> {
        \begin{block}{Trois étapes}
            \begin{enumerate}
                \item <5-> Question initiale sur un phénomène
                \item <5-> Développement d'une théorie proposant une réponse à la question
                \item <5-> Collecte des données empiriques pour tester la théorie
                    \begin{itemize}
                        \item Les données sont collectés de manière systématique!
                    \end{itemize}
            \end{enumerate}
        \end{block}
    }
    \onslide<6-> La méthode scientifique s'ancre dans le courant \textbf{positiviste}!
\end{frame}

\begin{frame}{Le positivisme}
    \onslide<1-> {
        \begin{columns}
            \begin{column}{.7\textwidth}
                \begin{itemize}
                    \item Introduit par Auguste Compte
                    \item Les phénomènes peuvent être étudiés en les observant.
                    \item Ces observations peuvent être regroupées afin d'obtenir des faits et créer théories qui peuvent nous aider à comprendre le monde.
                \end{itemize}
            \end{column}
            \begin{column}{.3\textwidth}
            \includegraphics[width=\textwidth]{comte}
            \end{column}
        \end{columns}
    }
\end{frame}


\begin{frame}{C'est quoi le positivisme?}
        \begin{columns}
            \begin{column}{.7\textwidth}
                \onslide<2-> {
                \begin{block}{Positif?}
                    \onslide<3-> {
                    Une théorie positiviste est une théorie qui est objective et basée sur des faits.
                }
                \end{block}
            }
                \onslide<4-> {
                \begin{block}{Normatif}
                \onslide<4-> {
                    Une théorie normative est une théorie qui est basée sur des jugements de valeurs et des interprétations.
                }
                \end{block}
        }
            \end{column}
            \begin{column}{.3\textwidth}
            \includegraphics[width=\textwidth]{glass}
            \end{column}
        \end{columns}
\end{frame}


\begin{frame}{La sociologie positiviste}

    L'étude de la société basé sur l'observation systématique de comportements sociaux.
    \begin{itemize}
        \item Importance de l'objectivité!
        \item Mettre de côté ses valeurs et croyances
        \item Approcher son objet d'étude comme un observateur neutre
        \item Se baser sur des preuves empiriques pour répondre à des questions sur le fonctionnement du monde social.
    \end{itemize}

    \begin{block}{Quel types de preuves empiriques?}
    \begin{itemize}
        \item Données quantitatives
        \item Données qualitatives
    \end{itemize}
    \end{block}
\end{frame}

\begin{frame}{La sociologie positiviste}
    \begin{block}{La recherche quantitative}
        La recherche quantitative se concentre sur l'étude de phénomènes sociaux observables dans le monde en utilisant généralement des données quantifiables et des modèles statistiques.
    \begin{itemize}
        \item L'objectif est de mesurer
        \item Données sous forme numérique
        \item Les informations peuvent être comptées \\(âge, revenu, ... mais aussi sexe, département...)
    \end{itemize}
    \end{block}

    On s'intéresse à mesurer l'effet d'une variable sur une autre:
    \begin{itemize}
        \item $Y = \alpha + \beta*x_i + \epsilon$
    \end{itemize}
\end{frame}

\begin{frame}{La sociologie positiviste}
    \begin{block}{La recherche qualitative}
        La recherche qualitative est l'étude de phénomènes sociaux observables plus difficilement quantifiables en se basant généralement sur des entretiens, questionnaires ou de l'observation participante ou non.
    \begin{itemize}
        \item L'objectif est d'illustrer ou de caractériser
        \item Données ne sont pas sous forme numérique
        \item L'information dont on a besoin ne peut être ou ne devrait pas être simplifié a un nombre dans un tableau
    \end{itemize}
    \end{block}

    On s'intéresse aux opinions et au valeurs latentes:
    \begin{itemize}
        \item Pourquoi certains citoyens refusent d'aller voter?
    \end{itemize}
\end{frame}

\begin{frame}{La sociologie positiviste}
    \begin{block}{Limites}
    Les études scientifiques impliquant des sujets humains et leur comportements viennent avec de nombreux défis.
    \begin{itemize}
        \item Ce qui nous intéresse n'est pas toujours observable
        \item Les humains ne sont pas toujours prévisibles
        \item Expériences difficilement contrôlables
        \item Implications éthiques
        \item On ne veut pas forcément modifier les conditions réelles
    \end{itemize}
    \end{block}
\end{frame}

\begin{frame}{La sociologie positiviste}
    \begin{columns}
        \begin{column}{0.7\textwidth}
    \begin{block}{Hawtorne Effect}
    \begin{itemize}
        \item Elton Mayo,
        \item Objectif : améliorer la productivité des travailleurs.
        \item Groupe contrôle : mêmes conditions
        \item Groupe test : ajustement de l'éclairage, heures de travail, pauses, ...
    \end{itemize}
    \end{block}
        \end{column}
        \begin{column}{0.3\textwidth}
            \includegraphics[width=\textwidth]{light}
        \end{column}
    \end{columns}
    \only<2>{Les travailleurs du groupe test sont devenus plus productifs et les taux d'absentéisme ont diminué!}
    \only<3>{La croissance de productivité observée n'est pas due aux facteurs expérimentaux testés!}
    \only<4>{Les chercheurs ont pris conscience que leur présence affecte les résultats de l'expérience (biais dans la collecte de données).\\ -> \textbf{Hawtorne effect} et illustre l'importance des entretiens!}
\end{frame}


\begin{frame}{La sociologie positiviste}
    \begin{columns}
        \begin{column}{0.8\textwidth}
    \begin{itemize}
        \item Tous les phénomènes sociaux ne sont applicables à tous les individus à toutes les périodes.
    \onslide<2->{
        \item La vérité n'est pas toujours objective!
    }
    \end{itemize}
        \end{column}
    \onslide<3->{
        \begin{column}{0.4\textwidth}
            \includegraphics[width=\textwidth]{ramen}
        \end{column}
    }
    \end{columns}
\end{frame}

\begin{frame}{La sociologie positiviste}
    \begin{block}{Autres considérations}
    \end{block}
    \begin{itemize}
        \item Les expériences subjectives sont également valides, importantes et peuvent aussi être étudiés.
        \item Nous ne pouvons pas généraliser cela à l'ensemble des individus et trouver la vérité avec un grand V.
        \item Mais nous pouvons nous intéresser à comment différentes tendances émergeant des différentes expériences subjectives individuelles pour former les structures qui composent notre monde social.
    \end{itemize}

\textbf{La subjectivité} c'est la signification que les individus donnent à leur expériences vécues.

\end{frame}

\section{Le projet de recherche}

\begin{frame}{Le projet de recherche}
    \begin{block}{Exercice préparatoire}
    \end{block}
    \begin{enumerate}
        \item Identifiez un thème
        \item Recherchez des articles scientifiques
        \item Identifier une problématique à partir des articles
        \item Identifier quel types de données pourraient vous aider
        \item Tentez de collecter quelques données
    \end{enumerate}

\textbf{Présentez!}

\end{frame}

\section{Bonus}

\begin{frame}{Bonus...}
        \begin{itemize}
            \item[] \faSlack ~Slack
            \item[] \faGithubSquare~Github
        \end{itemize}
\end{frame}


\begin{frame}[standout]
    \begin{flushright}
        \vspace{100pt}
        \small\faTwitter~$@$mickaeltemporao
    \end{flushright}
\end{frame}

\end{document}
